\documentclass[letterpaper,titlepage,oneside]{report}
\usepackage[margin=1.0in]{geometry}
\usepackage{graphicx}
\usepackage{float}
\usepackage{caption}
\usepackage{subcaption}
\usepackage{multirow}
\usepackage{booktabs}
\usepackage{tabularx}
\usepackage{todonotes}
\usepackage{hyperref}
\usepackage{amsmath}
%% \usepackage{dcolumn}
%% \usepackage{amsfonts}
%% \usepackage{hyperref}
%% \usepackage{xcolor}
%% \setlength\parindent{0pt}

\begin{document}

\begin{titlepage}
  \centering
  \vspace*{\fill}
  {\huge\scshape ENPH 257 Lab Report\par}
  \vspace{1cm}
  {\large Brunette, Jacob\quad}
  {\large Fullerton, Dilyn\quad}
  {\large Watt, Ryan\quad}
  {\large Yao, Dickson\par}
  \vspace{1cm}
  {\large Instructor: Dr. Christopher Waltham\par}
  \vspace{1cm}
  {\large \today\par}
  \vspace*{\fill}
  \vfill
\end{titlepage}

\begin{abstract}
  hello world
  % TODO overview of the experiment and results (like Executive Summary)
\end{abstract}

\chapter{Introduction}
\label{ch:intro}

\section{Heat transfer: Conduction, convection, and radiation}
%% TODO theoretical background on heat transfer

\section{The convection-diffusion equation}
%% TODO the full equation, along with our approximations and considerations

\chapter{Objective}
\label{ch:objective}
%% TODO describe the goal of the experiment

\chapter{Method}
\label{ch:method}
%% TODO how did we do the experiment (such that someone else could reproduce it)

\section{Experimental setup}
%% TODO rod, circuit, thermocouples, etc

\subsection{Rod dimensions}
%% TODO

\subsection{Thermocouple circuit}
%% TODO

\section{Calibration}
%% TODO procedure for calibration of thermocouples

\section{Data aquisition}
%% TODO procedure for acquiring data

\chapter{Results}
\label{ch:results}
%% TODO

\chapter{Simulation and optimization}
\label{ch:simulation}

\section{One-dimensional explicit-step simulation}
%% TODO description of python simulation

\subsection{Finite difference equation}
%% TODO

\subsection{Implementation}
%% TODO

\section{Optimization of one-dimensional simulation}
%% TODO numerical optimization of simulation parameters

\section{High-level simulation}
%% TODO SolidWorks simulation

\chapter{Analysis}
\label{ch:analysis}
%% TODO quality of data, fits, errors, uncertainty, and other considerations

\chapter{Conclusion and Summary}
\label{ch:conclusion}
%% TODO final conclusion: what did we learn and how is it relevant?

\appendix

%% TODO figure out how to site sources (and do it)

\chapter{Comparison of Finite Differencing Methods}
\label{ch:finitediff}
The following are methods of finite differencing, demonstrated
using the simple example of the one-dimensional heat conduction equation
\begin{equation}\label{eq:condsimp}
  \frac{\partial}{\partial t}u(x,t) = \alpha\frac{\partial^2}{\partial x^2}u(x,t).
\end{equation}

\section{Explicit method}

\subsection*{Finite difference equation}
In the explicit method for finite differencing, the temperature for
each spatial point can be independently determined from the previous
step. That is,
\begin{equation}\label{eq:explicit}
  u_i^{n+1} = u_i^n + \alpha\cdot\frac{\delta t}{\delta x^2}\cdot
  \left(u_{i-1}^n - 2u_i^n + u_{i+1}^n\right),
\end{equation}
where the superscript $n \in [0, T]$ signifies the step in time
$t_n$ and the
subscript $i \in [0, X]$ represents the position
$x_i$. $\delta t$ and $\delta x$
are the time steps in time and position respectively.
\\\\
If we define
\begin{equation*}
  \zeta = \alpha\cdot\frac{\delta t}{\delta x^2},
\end{equation*}
then this can be written as a linear relationship
\begin{equation}\label{eq:explicitlinear}
  \mathbf{u}^{n+1} = \mathbf{D}\mathbf{u}^{n},
\end{equation}
where $\mathbf{D}$ is the matrix defined by
\begin{equation}\label{eq:explicitdmatrix}
  \mathbf{D}_{ij} = 
  \begin{cases}
    \text{BC}, & \quad i=0 \text{ or } i=X \\
    \zeta,     & \quad 0<i<X \text{ and } |i-j|=1 \\
    1-2\zeta,  & \quad 0<i<X \text{ and } i-j=0 \\
    0,         & \quad \text{else} \\
  \end{cases}
\end{equation}
The top and bottom rows ($i=0$ and $i=X$) are determined from
boundary conditions. Note that as $\zeta$ approaches $0$,
$\mathbf{D}$ approaches the identity matrix. Thus $\zeta$ should be
as small as possible for the best approximtation. The condition for
stability is
\begin{equation*}
  0<\zeta<1.
\end{equation*}

\subsection*{Error in time}
The explicit method uses a forward difference in time. The
expression for the derivative approximation is derived from
the power series expansion in $\delta t$
\begin{equation}\label{eq:explicitexpansiondt}
  u(x,t+\delta t) = u(x,t) + \frac{\partial}{\partial t}u(x,t)\cdot\delta t
  + \frac{\partial^2}{\partial t^2}u(x,t)\cdot\frac{\delta t^2}{2!} + \dots
\end{equation}
Thus we have an approximation for the time derivative
\begin{equation}\label{eq:expliciterrdt}
  \frac{u(x,t+\delta t)-u(x,t)}{\delta t} = \frac{\partial}{\partial t}u(x,t)
  + O(\delta t).
\end{equation}
The error in time due to this approximation is linear in $\delta t$.

\subsection*{Error in position}
The explicit method uses a central second derivative in space. The
expression for the second derivative approximation is derived from
the power series expansion in $\delta x$.
\begin{equation}\label{eq:explicitexpansiondx}
  u(x+\delta x,t) = u(x,t) + \frac{\partial}{\partial x}u(x,t)\cdot\delta x +
  \frac{\partial^2}{\partial x^2}u(x,t)\cdot\frac{\delta x^2}{2!} +
  \frac{\partial^3}{\partial x^3}u(x,t)\cdot\frac{\delta x^3}{3!}
  + \dots
\end{equation}
At $-\delta x$, this is
\begin{equation*}
  u(x-\delta x,t) = u(x,t) - \frac{\partial}{\partial x}u(x,t)\cdot\delta x +
  \frac{\partial^2}{\partial x^2}u(x,t)\cdot\frac{\delta x^2}{2!} -
  \frac{\partial^3}{\partial x^3}u(x,t)\cdot\frac{\delta x^3}{3!}
  + \dots
\end{equation*}
Thus the approximation for the second spatial derivative is
\begin{equation}\label{eq:expliciterrdx}
  \frac{u(x-\delta x,t)-2u(x,t)+u(x+\delta x,t)}{\delta x^2} =
  \frac{\partial^2}{\partial x^2}u(x,t) + O(\delta x^2).
\end{equation}
The error in position due to this approximation is quadratic in $\delta x$.

\section{Implicit method}
%% TODO

\section{Crank-Nicolson method}
%% TODO

\end{document}
